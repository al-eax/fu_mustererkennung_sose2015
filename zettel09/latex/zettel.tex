\documentclass[a4paper,10pt]{article}
\usepackage[utf8]{inputenc}
\usepackage{amsmath}
\usepackage{amssymb}
\usepackage{graphicx}
%für quellcode
\usepackage{listings} 

\newcommand{\image}[3]{\begin{figure}[h]
                            \includegraphics[#1]{#3}
                            \caption{#2}
                       \end{figure}
                       }

\lstset{
    language = octave
}
%begin{lstlisting}

\begin{document}
%########################################
%		Titel
%########################################
\begin{center}
    \section*{Mustererkennung Übungszettel}
     \today
\end{center}
$ $
\newline
\begin{tabular}{r|l l}
    Name & Alexander Hinze-Hüttl & Kevin Pandura\\
    Matrikelnummer & 4578322 & 4562742\\
\end{tabular}
\newline
$ $
\newline
\newline
%########################################
%		Ende
%########################################

\section*{Aufgabe 1}

Lassen unser Netz $50000$ mal trainieren. Als Schrittweite hat sich $-0.1$ als zuverlässig herausgestellt.\\


Mit den Gewichtsmatrizen $\overline{W_1} =    \left(\begin{matrix}
0.0068625 & 0.6252486\\
0.2519384 & 0.6537852\\
0.3643390 & 0.8197155
\end{matrix} \right)$ , $\overline{W_2} =    \left(\begin{matrix}
0.16418\\
0.32411\\
0.91243
\end{matrix} \right)$ 
Erhalten wir den Output:\\
\texttt{
1. expected 0      predicted 0\\
2. expected 1      predicted 1\\
3. expected 1      predicted 1\\
4. expected 0      predicted 0
}
\\
\subsection*{Code}
\begin{lstlisting}
0;
% [a,b,c] -> [a,b,c,1]
function e = add1(o)
  [h,w] = size(o);
  e = ones(h,w + 1);
  e(1,1:w) = o(1:w);  
endfunction

% sigmoide Funktion
function s = sig(x)
  s = 1.0/(1.0  + exp(-x));
endfunction

%Erstellt zwei zufaellige Matrizen
function [W1,W2] = createRanwomW1W2(m,n,k)
  W1 = rand(n + 1,k);
  W2 = rand(k + 1,m);
endfunction

%Berechnet Diagonalmatrizen mit Ableitungen
function [D1,D2] = createD1D2(o1,o2,m,k)
  D1 = zeros(k,k);
  l = 1;
  for x = 1:k
    D1(x,x) = o1(1,x) * (1-o1(1,x));
  endfor

  D2 = zeros(m,m);

  for x = 1:m
    D2(x,x) = o2(1,x) * (1 - o2(1,x));
  endfor
endfunction

%Neuronales Netz sagt einen Wert vorher
function [r1,r2] = net(input, W1, W2, m, n , k)
  ot = add1(input);
  o11 = ot * W1;

  o1 = o11;
  for i = 1:k
    o1(1,i) = sig(o11(1,i));
  endfor
  
  o2 = add1(o1) * W2;
  for i = 1:m
    o2(1,i) = sig(o2(1,i));
  endfor
  r1 = o1;
  r2 = o2;
endfunction


n = 2; %input
k = 2; %hiddeen units
m = 1; %output units

trainData = [0,0,0;
             1,0,1;
             0,1,1;
             1,1,0];
lines = size(trainData,1);

[W1,W2] = createRanwomW1W2(m,n,k)
for i = 1:50000
   
    currentLine = trainData(mod(i,4)+1,:);
    expected = currentLine(1,3);
    [o1,o2] = net(currentLine(1,1:2),W1,W2,m,n,k);
    [D1,D2] = createD1D2(o1,o2,m,k);
    W22 = W2(1:k,1:m);
    e = o2 - (expected);
    
    delta2 = D2*transpose(e);
    delta1 = D1*W22*delta2;
    
    dW2T = -0.1*delta2*add1(o1);
    dW1T = -0.1*delta1*add1(currentLine(1,1:2));

    W1 = W1 + transpose(dW1T);
    W2 = W2 + transpose(dW2T);
endfor

for i = 1:lines
  currentLine = trainData(i,:);
  [o1,o2] = net(currentLine(1,1:2),W1,W2,m,n,k);
  expected = currentLine(1,3);
  predicted = round(o2);
  fprintf("%d. expected %d      predicted %d\n", i , expected, predicted);
endfor 

\end{lstlisting}

 \section*{Aufgabe 2}
 	Normieren zunächst alle Punkte mit der
 	Länge des längsten Punktvektors.
 	\\
 	Wandeln Digits in Vektoren um:\\ $0 = [1,0,0,0,0,0,0,0,0,0]\\1 = [0,1,0,0,0,0,0,0,0,0]\\...\\9 = [0,0,0,0,0,0,0,0,0,1]$\\
	und trainieren dann das Netz für $k = 2,4,8$.\\
	Wählen als Schrittweite $-1$, iterieren $4$ mal über den Trainingsdatensatz.
	\subsection*{k=8}
		Erkennungsrate = $0.91193$\\
		\begin{tabular}{||l|l|l|l|l|l|l|l|l|l|l|}
		 \hline 
			& 0 & 1 & 2 & 3 & 4 & 5 & 6 & 7 & 8 & 9 \\ \hline \hline
			0 & 738   &  5  &   0  &   0 &   13    & 5 &    1  &   1&    16 &     1\\ \hline
			1 &0&   693  &  18 &   35  &   4   &  0  &  11  &  10 &    0 &    8\\ \hline
			2 &0 &   29  & 701   &  0 &    3  &   0   &  0 &   47   &  0 &    0\\ \hline
			3 &0   &   3 &     2&    713 &     0&      0 &     0&      1 &     0 &     0\\ \hline
			4 &0    & 0&     1   &  1&   776  &   0  &   0   &  0   &  0    & 2\\ \hline
			5 &3 &    0  &   1 &  118 &   38 &  459 &   20 &    0 &    0  &  81\\ \hline
			6 &  1  &   0  &   0 &    2 &    2 &    1  & 713  &   0  &   1  &   0\\ \hline
			7 &0  &  62 &   12   &  3 &    5  &   0 &    0  & 693 &    1  &   2\\ \hline
			8 &13   &  13  &    0  &    0  &    0   &   4  &    0  &    9  &  680 &     0\\ \hline
			9 & 0&   40 &    1     & 6 &    2  &   1 &    0 &    0 &    1 &  668\\ \hline
		\end{tabular}
		
		\subsection*{k=4}
		Erkennungsrate = $0.86816$\\
		\begin{tabular}{||l|l|l|l|l|l|l|l|l|l|l|}
		 \hline 
			& 0 & 1 & 2 & 3 & 4 & 5 & 6 & 7 & 8 & 9 \\ \hline \hline
			0 & 720    &  0 &     0    &  0   &   5   &   5  &    2  &    0   &  43  &    5\\ \hline
			1 &  0  &  617   &  60  &   22  &    2  &    0   &   2   &  25  &   22   &  29\\ \hline
			2 &0    & 67  &  663   &   1  &    0   &   0  &    0    & 49   &   0   &   0\\ \hline
			3 &0  &   17   &   2  &  686    &  0  &    0   &   9  &    1   &   0  &    4\\ \hline
			4 &3   &   2   &   1  &    6  &  747  &    0   &   1   &   0   &   0   &  20\\ \hline
			5 &1   &   1  &    0   &  79   &   0   & 449  &   16   &   0   &  26  &  148\\ \hline
			6 & 0  &    0    &  0   &   0  &    1  &    1  &  716   &   0   &   2  &    0\\ \hline
			7 &0   &  86   &   0    &  0    &  0  &    1  &   34  &  645   &   5  &    7\\ \hline
			8 & 31  &    3   &   0  &    0   &   0  &    5 &    27   &  26  &  612  &   15\\ \hline
			9 &  0   &  18  &    0    & 14  &   10   &  10     & 0   &   0   &  16  &  651\\ \hline
		\end{tabular}
		
		\subsection*{k=2}
		Erkennungsrate = $0.66026$\\
		\begin{tabular}{||l|l|l|l|l|l|l|l|l|l|l|}
		 \hline 
			& 0 & 1 & 2 & 3 & 4 & 5 & 6 & 7 & 8 & 9 \\ \hline \hline
			0 & 757  &   0    & 0 &    0  &   5  &  12  &   0   &  0  &   3 &    3\\ \hline
			1 & 0 &  620&   108  &  21   &  0 &    0   & 10 &   18   &  0&     2\\ \hline
			2 & 0  &  52 &  711  &   0 &    0  &   0  &   0 &   17 &    0  &   0\\ \hline
			3 & 0  &  99   &  1 &  559 &   40 &    0  &   0  &  20  &   0 &    0\\ \hline
			4 & 109 &    1   &  0  &  44 &  337  &  45 &    8 &   14&     0 &  222\\ \hline
			5 & 92  &   9   &  0 &  139  &  71  & 307&    46  &  12  &  43 &    1\\ \hline
			6 & 1 &    0 &    0 &    0   &  0 &   11  & 708&     0 &    0  &   0\\ \hline
			7 & 1  & 101  &  15 &  185 &    6 &   23&    52 &  392 &    0 &    3\\ \hline
			8 &  259  &  24&     0  &  23  &  67  & 246&    51&     2  &  34 &   13\\ \hline
			9 & 14 &   59 &    0  &   1 &  122  &   0  &   0  &   0  &   0  & 523\\ \hline
		\end{tabular}
		
		\subsection*{Code}
		\begin{lstlisting}
0;
%macht [a,b,c] zu [a,b,c,1]
function e = add1(o)
  [h,w] = size(o);
  e = ones(h,w + 1);
  e(1,1:w) = o(1:w);  
endfunction

function s = sig(x)
  s = 1.0/(1.0  + exp(-x));
endfunction


function [W1,W2] = createRanwomW1W2(m,n,k)
  W1 = rand(n + 1,k);
  W2 = rand(k + 1,m);
endfunction

function [D1,D2] = createD1D2(o1,o2,m,k)
  D1 = zeros(k,k);
  l = 1;
  for x = 1:k
    D1(x,x) = o1(1,x) * (1-o1(1,x));
  endfor

  D2 = zeros(m,m);

  for x = 1:m
    D2(x,x) = o2(1,x) * (1 - o2(1,x));
  endfor
endfunction

function [r1,r2] = net(input, W1, W2, m, n , k)
  ot = add1(input);
  o11 = ot * W1;

  o1 = o11;
  for i = 1:k
    o1(1,i) = sig(o11(1,i));
  endfor
  
  o2 = add1(o1) * W2;
  for i = 1:m
    o2(1,i) = sig(o2(1,i));
  endfor
  r1 = o1;
  r2 = o2;
endfunction

function n = numToOutput(i)
  n = zeros(1,10);
  n(1,i+1) = 1;
endfunction

function n = outputToNum(o2)
  [x,y] = max(o2);
  n = y-1;
endfunction

function n = normDigit(d)
  points = zeros(8,2);
  points(1,:) = d(1,1:2);
  points(2,:) = d(1,3:4);
  points(3,:) = d(1,5:6);
  points(4,:) = d(1,7:8);
  points(5,:) = d(1,9:10);
  points(6,:) = d(1,11:12);
  points(7,:) = d(1,13:14);
  points(8,:) = d(1,15:16);
  m = norm(points(1,:));
  maxIndex = 1;
  for i = 1: size(points,1)
    if norm(points(i,:)) > m
      maxIndex = i;
      m = norm(points(i,:));
    endif
  endfor
  
  n = d / m;
endfunction


data = load("/home/alex/fu_mustererkennung_sose2015/zettel09/pendigits-training.txt");
testData = load("/home/alex/fu_mustererkennung_sose2015/zettel09/pendigits-training.txt");
lines = size(data,1)

n = 16; %input
k = 2; %hiddeen units
m = 10; %output units



[W1,W2] = createRanwomW1W2(m,n,k)
for foo = 1:4
  for i = 1:lines
    currentLine = data(i,:);
    expected = currentLine(1,17);
    input = normDigit(currentLine(1,1:16));
    [o1,o2] = net(input,W1,W2,m,n,k);
    [D1,D2] = createD1D2(o1,o2,m,k);
    W22 = W2(1:k,1:m);
    e = o2 - numToOutput(expected);
    
    delta2 = D2*transpose(e);
    delta1 = D1*W22*delta2;
    
    dW2T = -1*delta2*add1(o1);
    dW1T = -1*delta1*add1(input);

    W1 = W1 + transpose(dW1T);
    W2 = W2 + transpose(dW2T);
  endfor
endfor

lines = size(testData,1);
good = 0;
mat = zeros(10,10);
for i = 1:lines
  currentLine = testData(i,:);
  [o1,o2] = net(normDigit(currentLine(1,1:16)),W1,W2,m,n,k);
  expected = currentLine(1,17);
  
  %fprintf("%d. expected %d      predicted %d\n", i , expected, outputToNum(o2));
  if expected == outputToNum(o2)
    good += 1;
  endif
  mat(expected+1 , outputToNum(o2)+1) += 1;
endfor
k
mat
good / lines
		\end{lstlisting}
		
\end{document}